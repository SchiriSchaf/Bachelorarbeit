%% appendix.tex
%%

%% ==============================
\Appendix
\label{ch:Appendix}
%% ==============================



\section{Technische Doku über die Benutzung von den MITK Segmentierungstools}

Als erstes muss in den DICOM Ordner der Ordner mit ungefähr 200 Elementen gefunden werden.

\begin{figure}[H] 
\centering 
\includegraphics[width=0.7\textwidth]{Logos/MITK_Doku/1.PNG}
\caption{1} 
\label{fig:eins} 
\end{figure}

Anschließend kann eine der Dateien des Ordners via drag and drop in den \textit{Data Manager} gezogen werden und das Fenster das erscheint bestätigt werden.
\newline
In der Mitte sind die verschiedenen Schnittbilder zu sehen, durch die mit dem Mausrad durchgegangen werden kann.
\newline
Die Zahlen von 1 bis 5 markieren verschiedenen Tools. Werden diese angeklickt, öffnet sich an der Seite oder unter den Schnittbildern das jeweilige Tool. Um ein Tool auf ein Volumen anzuweden, muss dieses immer im \textit{Data Manager} ausgewählt sein.
\newline
Mit dem Regler am rechten Rand der Schnittbilder kann die Darstellung der Grauwerte eingestellt werden.

\begin{figure}[H] 
\centering 
\includegraphics[width=\textwidth]{Logos/MITK_Doku/2.PNG}
\caption{2} 
\label{fig:zwei} 
\end{figure}

Das Tool mit der Nummer 1 macht es möglich eine Transferfunktion zu definieren. Anhand der Intensitätswerte, den Punkten über dem Diagramm und dem Farbstrahl darunter kann die Transferfunktion angepasst werden. Ist der Haken bei \textit{Volumerendering} gesetzt, so wird die Visualisierung im Fenster links unten angezeigt.

\begin{figure}[H] 
\centering 
\includegraphics[width=0.7\textwidth]{Logos/MITK_Doku/3.PNG}
\caption{3} 
\label{fig:drei} 
\end{figure}

Das Tool mit der Nummer 2 zeigt Statistiken zum Volumen an, wie zum Beispiel Verteilung der Intensitätswerte, der Median dieser etc.

\begin{figure}[H] 
\centering 
\includegraphics[width=0.4\textwidth]{Logos/MITK_Doku/4.PNG}
\caption{4} 
\label{fig:vier} 
\end{figure}

Das Tool Nummer 3  bietet verschiedene 2D und 3D Segmentierungstools an. Zunächst muss jedoch eine neue Segmentierung über den rot markierten Knopf erstellt werden.

\begin{figure}[H]
\centering 
\includegraphics[width=0.7\textwidth]{Logos/MITK_Doku/5.PNG}
\caption{5} 
\label{fig:fuenf} 
\end{figure}

Wurde dies getan, kann das 3D Tool \textit{Region Growing 3D} ausgewählt werden.

\begin{figure}[H] 
\centering 
\includegraphics[width=0.7\textwidth]{Logos/MITK_Doku/6.PNG}
\caption{6} 
\label{fig:sechs} 
\end{figure}

Anschließend kann mit Shift + Linke Maustaste eine Punkt im Volumen ausgewählt werden. Danach muss mit  \textit{Run Segmentation} die Segmentierung ausgeführt werden.

\begin{figure}[H] 
\centering 
\includegraphics[width=0.7\textwidth]{Logos/MITK_Doku/7.PNG}
\caption{7} 
\label{fig:sieben} 
\end{figure}

Nun kann der Threshhold angepasst und das Region Growing mit dem \textit{Adapt Region Growing} Regler angepasst werden. Wird das gezielte Ergebnis erwünscht, kann es mit \textit{Confirm Segmentation} bestätigt werden.

\begin{figure}[H] 
\centering 
\includegraphics[width=0.7\textwidth]{Logos/MITK_Doku/8.PNG}
\caption{8} 
\label{fig:acht} 
\end{figure}

Mit Tool Nummer 4 kann das Ergebnis mit verschiedenen Operationen verbessert werden. Hierbei wird oft \textit{Closing} benutzt. Anschließend kann mit Rechtsklick auf die Segmentierung im \textit{Data Manager} der Befehl \textit{Create smoothed polygon model} ausgeführt werden

\begin{figure}[H] 
\centering 
\includegraphics[width=0.7\textwidth]{Logos/MITK_Doku/9.PNG}
\caption{9} 
\label{fig:neun} 
\end{figure}

Nachdem alle diese Schritte befolgt wurden ist im linken Unteren Kasten die Segmentierung sehen. Mit Tool Nummer 5 kann die Darstellung der Segmentierung verändert werden.

\begin{figure}[H] 
\centering 
\includegraphics[width=0.7\textwidth]{Logos/MITK_Doku/10.PNG}
\caption{10} 
\label{fig:zehn} 
\end{figure}