\Abstract{
Die Visualisierung von medizinischen Daten wird immer wichtiger. Im Rahmen des HoloMed Projektes wird versucht einen Arzt bei einer Ventrikelpunktion, einem sehr präzise auszuführender neurochirurgischer Eingriff, zu unterstützen. Dazu soll ihm das Ventrikelystem mithilfe einer AR-Brille angezeigt werden. Daraus ergibt sich die Aufgabe das System anhand von CT-Daten segmentieren zu können. Für diese Aufgabe eignet sich eine Transferfunktion. Das Ziel dieser Arbeit ist die Implementierung einer Transferfunktion, um das Ventrikelsystem zu segmentieren und hervorzuheben. Dabei wurde sich für die Clustering-basierte Transferfunktion von Nguyen \cite{nguyen2012clustering} entschieden. Diese wendet ein zweistufiges Clustering an. Dabei werden zunächst die LH-Werte des Volumens berechnet und ein Histogramm erstellt. Die LH-Werte beschreiben die Grenzen von verschiedenen Materialien im Volumen. Auf den Histogramm findet anschließend der erste Clusteringschritt statt. Danach werden die entstandenen Cluster nochmals anhand ihrer räumlichen Informationen im Volumen geclustert. Die beiden Clusteringschritt benutzen dabei jeweils \textit{Meanshiftclustering}. Aus den entstehenden finalen Clustern kann das Ventrikelsystem aus mehreren berechneten Clustern zusammengesetzt werden. Die Implementierung kam in dieser Arbeit zu einem Erfolg und es war möglich die Ventrikelsysteme von gesunden Menschen zu segmentieren.

}
