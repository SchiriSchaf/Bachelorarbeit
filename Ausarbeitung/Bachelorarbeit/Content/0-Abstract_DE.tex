\Abstract{
Die Visualisierung von Daten spielt auch bei medizinischen Anwendungen eine wichtige Rolle. Im Rahmen des HoloMed Projektes wird ein Arzt bei einer Ventrikelpunktion, einem sehr präzise auszuführenden neurochirurgischen Eingriff, unterstützt, indem ihm das Ventrikelystem mithilfe einer AR-Brille angezeigt wird. Dabei entsteht die Aufgabe das System anhand von CT-Daten zu segmentieren. 
\newline
Im Rahmen dieser Arbeit wurde eine Transferfunktion implementiert, die das Ventrikelsystem segmentiert und hervorhebt. Dabei wurde das Clustering-basierte Verfahren von Nguyen \cite{nguyen2012clustering} verwendet. Dies wendet ein zweistufiges Clustering an. Dabei werden zunächst die LH-Werte des Volumens berechnet und ein Histogramm erstellt. Die LH-Werte beschreiben die Grenzen von verschiedenen Materialien im Volumen. Auf den Histogramm findet anschließend der erste Clusteringschritt statt. Danach werden die entstandenen Cluster nochmals anhand ihrer räumlichen Informationen im Volumen geclustert. Die beiden Clusteringschritt benutzen dabei jeweils \textit{Meanshiftclustering}. Aus den entstehenden finalen Clustern kann das Ventrikelsystem aus mehreren berechneten Clustern zusammengesetzt werden.
\newline
Die Implementierung im Rahmen dieser Arbeit war erfolgreich. Es ist nun möglich die Ventrikelsysteme von gesunden Menschen zu segmentieren.

}
