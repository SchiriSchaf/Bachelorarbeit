\Abstract{
Die Visualisierung von Daten spielt im medizinischen Kontext eine wichtige Rolle. Im Rahmen des HoloMed Projektes wird versucht einen Arzt bei einer Ventrikelpunktion, einem sehr präzise auszuführenden neurochirurgischen Eingriff, zu unterstützen. Dies soll geschehen, indem ihm das Ventrikelystem mithilfe einer AR-Brille angezeigt werden soll. Dadurch entsteht die Aufgabe das System anhand von CT-Daten segmentieren zu können. Das Ziel dieser Arbeit ist die Implementierung einer Transferfunktion, die das Ventrikelsystem segmentiert und hervorhebt. Dabei wurde sich für das Clustering-basierte Verfahren von Nguyen \cite{nguyen2012clustering} entschieden. Diese wendet ein zweistufiges Clustering an. Dabei werden zunächst die LH-Werte des Volumens berechnet und ein Histogramm erstellt. Die LH-Werte beschreiben die Grenzen von verschiedenen Materialien im Volumen. Auf den Histogramm findet anschließend der erste Clusteringschritt statt. Danach werden die entstandenen Cluster nochmals anhand ihrer räumlichen Informationen im Volumen geclustert. Die beiden Clusteringschritt benutzen dabei jeweils \textit{Meanshiftclustering}. Aus den entstehenden finalen Clustern kann das Ventrikelsystem aus mehreren berechneten Clustern zusammengesetzt werden. Die Implementierung kam in dieser Arbeit zu einem Erfolg und es war möglich die Ventrikelsysteme von gesunden Menschen zu segmentieren.

}
