%% ==============================
\chapter{\iflanguage{ngerman}{Fazit}{Discussion}}
\label{sec:discussion}
%% ==============================


Nachdem die Ergebnisse im vorherigen Kapitel gezeigt und evaluiert wurden, wird in diesem ein Fazit gezogen.
\newline
Das Ziel der Bachelorarbeit, war die Implementierung eines Verfahrens zur erfolgreiche Visualisierung des Ventrikelsystem basierend auf Volumendaten. Dieses Ziel wurde erfüllt, jedoch mit deutlichen Abstrichen.


Zunächst gilt es die Robustheit der Implementierung zu betrachten. Es konnte nur in 3 von 15 Fällen das Ventrikelsystem erfolgreich visualisiert werden. Zwar waren diese Daten meist von besonderen Arten, jedoch konnte auch bei den normalen Ventrikeln nur in 50\% der Fälle ein zufriedenstellendes Ergebnis erzielt werden. Weiterhin war es nur für bestimmte Auflösungen überhaupt möglich die Berechnung der Cluster durchzuführen.
\newline
Die Robustheit des Systems ist folglich verbesserungswürdig. Die Segmentierung funktioniert, jedoch in nur sehr wenigen Fällen.


In den Fällen, in denen die Visualisierung geglückt ist, waren die Ergebnisse von einer guten Qualität. Das Ventrikelsystem war deutlich zu sehen und als solches auch klar zu identifizieren.
\newline
Es fehlten zwar bei allen Visualisierungen der dritte und vierte Ventrikel, jedoch sind diese für die Ventrikelpunktion nicht von Nöten. Dem Arzt, dem die Visualisierung des Ventrikelsystems im Operationssaal assistieren soll, sollten womöglich nur die relevanten Informationen angezeigt werden, damit diese stets übersichtlich bleiben und ihn nicht im schlimmsten Fall zusätzlich verwirren.
\newline
Jedoch fehlten auch bei den Seitenventrikel teils kleine, teils große Stücke in der Visualisierung. Des Weiteren gab es viele Ausreißer, die vom interviewten Arzt als störend empfunden wurden.
\newline
Weiterhin sind die Ergebnisse nicht glatt, sondern haben in der hervorgehobenen Struktur kleine Löcher und Unreinheiten.
\newline
Für die Zeit die für die Implementierung zur Verfügung stand, sind die Ergebnisse bereits solide. Sie benötigen jedoch an mehreren Stellen einer Verbesserung und müssten um weiter Verfahren erweitert werden, um ein sehr gutes Ergebnis zu erzielen.


Die Schritte, die vom Benutzer ausgeführt werden müssen, um zu einem Ergebnis zu gelangen, sind wenig intuitiv. Er benötigt Wissen darüber welche Befehle wann im Helper ausgeführt werden müssen und deren Syntax.
\newline
Weiterhin muss er selbst Dateien in speziellen Formaten speichern und laden. Außerdem muss der Anwender die passenden IDs manuell finden, was zeitaufwändig sein kann.
\newline
Im Bezug auf die Nutzerfreundlichkeit benötigt das Verfahren viel Überarbeitung, da es bisher schwer zu bedienen ist.
\newline
Dies ist aktuell jedoch der Fall, aus dem Grund, dass das Funktionieren der Methode oberste Priorität beim Programmierung hatte. Aus der schon mehrfach erwähnten Zeitknappheit wurde die Benutzerfreundlichkeit deshalb zunächst vernachlässigt.


Schließlich wird noch die Verarbeitungsgeschwindigkeit des Verfahren betrachtet.
\newline
Diese ist für die kleineren Auflösungen relativ schnell. Die benötigte Zeit zur Berechnung der Cluster wächst jedoch nicht linear mit der Größe der Eingabe.
\newline
Hierbei ist jedoch unklar, ob dies an der Implementierung oder an dem Verfahren an sich liegt, da über die Laufzeit der Meanshiftclustering keine Informationen vorliegen.


Zusammenfassend lässt sich sagen, dass die Bachelorarbeit zu einem positiven Ergebnis gelangt ist. Zwar gibt es noch viele Stellen, an denen Verbesserungen vorgenommen werden können, aber es war möglich die, im Kontext des HoloMed Projektes, wichtigen Bereiche des Ventrikelsystems zu segmentieren und hervorzuheben.
\newline
Wenn die eben erwähnten Probleme verbessert werden, wird es voraussichtlich mit diesem clusteringbasierten Verfahren konsistent möglich sein, gute Segmentierungen des Ventrikelsystems zu berechnen.
\newline
Ein Ausblick wie die verschiedenen genannten Probleme behoben und die Implementierung an vielen Stellen verbessert werden kann wird im nächsten Kapitel gegeben.

























