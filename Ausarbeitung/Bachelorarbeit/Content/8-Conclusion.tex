%% ==============================
\chapter{\iflanguage{ngerman}{Ausblick}{Conclusion}}
\label{sec:conclusion}
%% ==============================



Die Implementierung eines Clusteringbasierten Verfahren kam in dieser Arbeit zu einem Erfolg, jedoch gibt es noch sehr viele Verbesserungs- und Erweiterungsmöglichkeiten. Diese werden im folgenden Kapitel vorgestellt.


Die Visualisierung des Ventrikelsystems funktionierte bei nur wenigen Datensätzen. Bei den restlichen wurde das System aus verschiedenen Gründen nicht erkannt.
\newline
Es gab Probleme, wenn einerseits das Ventrikelsystem zu klein oder verformt war, oder andererseits die Hirnmasse im Kopf und um den Ventrikel herum andere Intensitätswerte als üblich hatten.
\newline
Bei normalen Ventrikel funktionierte es teilweise auch nicht. Diese hohe Inkonsistenz gilt es zu beheben. Dazu können verschiedene Veränderungen vorgenommen werden.
\newline
Da das Verfahren oft daran scheiterte, die Kanten von Ventrikeln zu erkennen, da sie zu klein oder der Unterschied zur Umgebung zu gering ist, wäre es eine Möglichkeit, einen noch kleinere Bandweite und damit ein noch genaueres Clustern über dem LH-Raum zu implementieren.
\newline
Dadurch würden jedoch noch mehr Cluster entstehen, was die Benutzung noch schwieriger macht.
\newline
Dies könnte jedoch behoben werden, indem einerseits Cluster die am Rande des Volumens liegen weggelassen werden. Da das Ventrikelsystem in der Mitte des Kopfes liegt, ist es klar, dass diese Cluster keine Teile eines Ventrikels enthalten können.
\newline
Andererseits könnte der LH-Wertebereich über dem geclustert wird noch kleiner gemacht werden. Wie in \autoref{fig:unity_s} und \autoref{fig:unity_u} zu sehen ist, werden die Seitenventrikel in einem Intensitätsbereich von 1025 bis 1030 bereits erkannt.
\newline
Allgemein würde ein weiteres Anpassen der Parameter auf das Ventrikelsystem zu besseren Ergebnissen führen.


Ein weiteres Problem war die Benutzerfreundlichkeit. Ein Anwender muss einen genauen Ablauf befolgen um zu einem Ergebnis zu kommen, indem er erst mithilfe der Konsole Daten laden, falls sie noch im .nrrd Format vorliegen formatieren, die Cluster erstellen lassen, diese in Unity laden, die passenden Cluster manuell herausfinden, und schließlich mit einem weiteren Befehl über die Kommandozeile zu einem Ergebnis verschmelzen lassen. Dieser Ablauf ist wenig Intuitiv und kann wie die Nutzerstudie aus Kapitel 6 gezeigt hat für Menschen ohne Programmiererfahrung zu einer Hürde werden.
\newline
Dies könnte an vielen Stellen verbessert werden. Beispielsweise können die Befehle der Konsole vereinfacht werden, sodass der Benutzer das notwendige $u$ sowie das Suffix .bin.txt nicht mehr angeben muss, da das Programm automatisch das Volumen umwandelt und im richtigen Dateiformat speichert.
\newline
Eine andere Idee wäre es, die zwei Programme zu einem zu verschmelzen, sodass der Anwender direkt in Unity das Volumen laden und die IDs berechnen lassen könnte.
\newline
Des Weiteren wäre es dann möglich, dass die Cluster direkt in Unity angezeigt, ausgewählt und verschmolzen werden könnten. Dadurch bestünde auch die Möglichkeit zwischen der Cluster und der finalen Ansicht hin und her zu wechseln.
\newline
Der Anwender muss sich dadurch nicht mehr mit Dateipfaden dem Speichern und Einlesen von Dateien im richtigen Format beschäftigen.
\newline
Weiterhin könnte Nutzung durch das Erstellen einer GUI intuitiver gemacht werden, damit keine Eingabe von Befehlen in eine Kommandozeile mehr nötig ist.


Ein Problem, dass auch vom Arzt erkannt wurde, ist die nicht glatte Darstellung im Ergebnis, sowie die vielen existierenden Ausreißer. Dies macht die Visualisierung teilweise sehr ungenau.
\newline
Dies lies sich lösen, indem auf das Endergebnis weitere Algorithmen zur Verbesserung der Segmentierung angewendet werden.
\newline
Beispielsweise könnten mit Dilatations- und Erosionsfiltern die Oberfläche geschlossen und kleine Ausreißer eliminiert werden.
\newline
Eine weitere Verbesserung hierbei wäre, einen Regiongrowingalgorithmus am Ende des Verfahrens anzuwenden. Dieser könnte das Ergebnis deutlich verbessern, da dadurch ein vollständigeres Bild des Ventrikelsystems entstehen und eventuell sogar weitere Ventrikel, wie zum Beispiel das Dritte und Vierte, erkannt werden könnten.


Weiterhin könnte an vermutlich mehreren Stellen des aktuellen Codes kleine Optimierungen vorgenommen werden.
\newline
Beispielsweise die Verbesserung der Suche nach Schleifen beim Berechnen der High-Werte, die im Kapitel \nameref{sec:implementation} beschrieben wurde.
\newline
Oder es könnte beim räumlichen Clustern ein Verschmelzen der im gleichen LH-Cluster gefundenen räumlichen Cluster hinzugefügt werden.
\newline
Dies sind nur zwei Beispiele und es gibt vermutlich mehrere Stellen an denen diese kleinen Verbesserungen vorgenommen werden könnten.
\newline
Diese Änderungen hätten möglicherweise nur geringe Auswirkungen, sie  könnten jedoch nützlich sein, um ein optimales Ergebnis zu erreichen.


Ein weiterer Punkt für Verbesserungen ist das Beheben der Bugs. Der Grund für das scheitern der Berechnung des Verfahrens auf dem ganzen, sowie auf den viertel Volumen muss gefunden und behoben werden.


Des Weiteren könnte getestet werden ob die Berechnungszeit noch weiter zu verbessern ist. Aktuell laufen alle Kalkulationen auf der CPU. Die Clusteringschritte könnte jedoch auf der GPU deutlich schneller berechnet werden.


