%% ==============================
\chapter{\iflanguage{ngerman}{Einleitung}{Introduction}}
\label{sec:Introduction}
%% ==============================



Computergestützte Verfahren werden in der Medizin immer wichtiger. Sie erleichtern dem Arzt seine Arbeit und senken die Wahrscheinlichkeit, dass bei einer Operation ein Fehler gemacht wird.
\newline
\todo{genauer Gründe Punktion beschreiben}
Ein Routineprozedur der Neurochirurgie ist die Punktion des Ventrikelsystems zur Drainage von Liquor. Diese wird häufig Nötig, wenn ein Patient beispielsweise unter einer Gehirnblutung, einem Schädelhirntrauma oder einem Schlaganfall leidet. Um die Punktion durchzuführen, muss der Chirurg eine Bohrlochtrepanation am sogenannten Kocherpunkt durchführen. Der Arzt muss anhand äußerer anatomischer Landmarker diesen Punkt auf wenige Zentimeter genau finden und die Stichrichtung der Punktion ausmachen. Dieses Verfahren ist sehr fehleranfällig. So kommt es nur in zwei drittel aller Operationen zu einem optimalen Ergebnis, wofür oftmals mehrfach punktiert werden muss.
\newline
Im Rahmen des HoloMed Projektes wird daran gearbeitet den Chirurg bei diesem Eingriff zu unterstützen. Der Plan ist es, dass der Arzt mithilfe einer AR-Brille angezeigt bekommt wo sich das Ventrikelsystem befindet und somit mit einer niedrigeren Fehlerwahrscheinlichkeit die Operation durchführen kann.
\newline
Dazu muss anhand der CT-Daten des Gehirns das Ventrikelsystem hervorgehoben und visualisiert werden. Um diese Aufgabe zu lösen eignet sich eine Übergangsfunktion, auch Transferfunktion genannt. Allgemein gesprochen ordnet eine Transferfunktion volumetrischen Daten optische Eigenschaften zu. Diese Aufgabe teilt sich in zwei Bereiche auf. Zum einen entscheidet die Funktion welche Daten gezeigt werden und zum anderen wie diese dargestellt werden, beispielsweise welche Farb- und Okklusionswerte diese erhalten. Dabei ist der erste Teil des wesentlich wichtigere.
\newline

























\todo{labels verweisen mit nameref}
Der Rest der Arbeit teil sich in folgendermaßen auf. In Kapitel 2 wird ein Überblick zu dem aktuellen Stand der Wissenschaft und Technik gegeben und Kapitel 3 beschäftigt sich mit der in dieser Arbeit verwendeten Methode. In Kapitel 4 wird das Softwaredesign der Implementierung erklärt und in Kapitel 5 die Implementierung besprochen. Kapitel 6 zeigt die Ergebnisse des Verfahrens und evaluiert diese. In Kapitel 7 wird ein Fazit gezogen und in Kapitel 8 ein Ausblick auf mögliche Verbesserungen gegeben.