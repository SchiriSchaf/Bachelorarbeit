%% Choose language: english or german
%% Choose Thesis type: seminar, bachelor, master, techreport
\documentclass[german,techreport]{IPRthesis}

%% ---------------------------------
%% | Information about the thesis  |
%% ---------------------------------
% TODO: Change this \titleenglish \titlegerman. Same for keywords.
\title{Implementierung eines Clustering-basierten Verfahrens zur Visualisierung von Volumenmodellen}
\titleotherlanguage{Implementierung eines Clustering-basierten Verfahrens zur Visualisierung von Volumenmodellen}

\author{Lukas Diewald}

\keywords{HoloMed, Clusteringbasiertes Verfahren, Implementierung}
\keywordsotherlanguge{HoloMed, Clusteringbasiertes Verfahren, Implementierung}


\reviewerone{Prof. Dr.-Ing. habil. Björn Hein}
\reviewertwo{Prof. Dr.-Ing. Torsten Kroeger}
%
% %% The advisors are PhDs or Postdocs
\advisorone{Christian Kunz M.Sc.}

%
% %% Please enter the start end end time of your thesis
\editingtime{xx. Month 2018}{xx. Month 2018}

%% --------------------------------
%% | Settings for word separation |
%% --------------------------------
% Help for separation:
% In german package the following hints are additionally available:
% "- = Additional separation
% "| = Suppress ligation and possible separation (e.g. Schaf"|fell)
% "~ = Hyphenation without separation (e.g. bergauf und "~ab)
% "= = Hyphenation with separation before and after
% "" = Separation without a hyphenation (e.g. und/""oder)

% Describe separation hints here:
\hyphenation{
% Pro-to-koll-in-stan-zen
% Ma-na-ge-ment  Netz-werk-ele-men-ten
% Netz-werk Netz-werk-re-ser-vie-rung
% Netz-werk-adap-ter Fein-ju-stier-ung
% Da-ten-strom-spe-zi-fi-ka-tion Pa-ket-rumpf
% Kon-troll-in-stanz
}


%% ------------------------
%% |    Including files   |
%% ------------------------
% Only files listed here will be included!
% Userful command for partially translating the document (for bug-fixing e.g.)
% \includeonly{%
% Content/0-Declaration,
% Content/0-Abstract_EN,
% Content/0-Abstract_DE,
% Content/1-Introduction,
% Content/2-State-of-the-art,
% Content/3-Methods,
% Content/4-Concept,
% Content/5-Implementation,
% Content/6-Results,
% Content/7-Discussion,
% Content/8-Conclusion,
% Content/11-Appendix,
% }

\settitle
%%%%%%%%%%%%%%%%%%%%%%%%%%%%%%%%%
%% Here, main documents begins %%
%%%%%%%%%%%%%%%%%%%%%%%%%%%%%%%%%
\begin{document}

%% Set PDF metadata
\setpdf

%% Set the title
\maketitle



\includedeclaration

\includeacknowledgments

\setcounter{page}{1}
\pagenumbering{roman}

%% ----------------
%% |   Abstract   |
%% ----------------
%% An abstract both in English
%% and German is mandatory.
%%
%% The text is included from the following files:
%% - Content/0-Abstract_EN
%% - Content/0-Abstract_DE
\includeabstract

%% ------------------------
%% |   Table of Contents  |
%% ------------------------
\inculdetableofcontents

\makenomenclature

%% -----------------
%% |   Main part   |
%% -----------------

\setcounter{page}{1}
\pagenumbering{arabic}
%% ==============================
\chapter{\iflanguage{ngerman}{Einleitung}{Introduction}}
\label{sec:Introduction}
%% ==============================


-transferfunktionen was ist das?
-ventrikelsystem
-medizinische bildverarbeitung
-ct daten



\todo{labels verweisen mit nameref}
Der Rest der Arbeit teil sich in folgendermaßen auf. In Kapitel
%% ==============================
\chapter{\iflanguage{ngerman}{Stand der Wissenschenschaft und Technik}{State of the art}}
\label{sec:state_of_the_art}
%% ==============================




Im Paper von Hong \cite{hong2003method} wird ein Approximationsbasiertes Verfahren zur Berechnung von Gradienten eines Volumens vorgestellt. 
\newline
Hierbei ist zu beachten, dass der Gradient nicht für einen Voxel direkt berechnet werden kann. Der Gradient liegt im Falle eines dreidimensionalen Volumens im Zentrum eins Würfels, der von 4 benachbarten Voxeln aufgepsannt wird. In Hongs Verfahren wird zur Berechnung die lokale 4x4x4 Nachbarschaft hinzugezogen.
\newline
\includegraphics[width=\textwidth]{Logos/VoxelEdges.PNG}
\todo{richtig bild zitieren u. evtl kleiner}

Die Funktionen für die Intensitätswerte wird im Paper mit:  $f(x,y,z) = Ax^{2}+By^{2}+Cz^{2}+2Fyz+2Gzx+2Hxy+2Ix+2Jy+2Kz+D$  approximiert. Den dreidimensionalen Gradientenvektor n erhält man, indem man die Funktion ableitet: $n = (Ax+Gz+Hy+I, By+Fz+Hx+J, Cz + Fy + Gx + K)$ .
\newline
Um den Gradienten zu Berechnen müssen die Parameter A,B,C,E,F,G,H,I,J,K  berechnet werden. Dies geschieht mithilfe der Methode der kleinsten Quadrate.




Kindlmann und Durkin  \cite{kindlmann1998semi} - gradient


Bajaj Countour: \cite{bajaj1997contour} -1d


Correa und Ma Transferfunktion,basierend auf Größe \cite{correa2008size}, occlusion \cite{correa2009occlusion}, visibility \cite{correa2009visibility}(später histogram\cite{correa2011visibility})


Größe: 

\todo{region growing}



Imagebased:

 
Wu and Qu
proposed a system that uses editing operations and stochastic
search of the transfer function parameters to maximize the
similarity between volume-rendered images given by the user

Wu und Qu schlugen in ihrer Arbeit vor \cite{wu2007interactive}


Das Gebiet der Transferfunktionen ist weit erforscht und es existieren bereits viele verschiedene Methoden und Herangehensweisen. In diesem Abschnitt wird ein Überblick über die unterschiedlichen Vorgehensweisen von Transferfunktionen gegeben. Dabei werden diese im Folgenden in die Kategorien: eindimensionale Transferfunktionen, zweidimensionale Transferfunktionen, mehrdimensionale Transferfunktionen, ... unterteilt. Hierbei ist jedoch zu beachten, das manche der hier vorgestellten Verfahren mehrere Kategorien verbinden.


\subsection{Eindimensionale Transferfunktionen}

Die einfachste Form der Transferfunktionrm, sind die eindimensionalen Transferfunktionen. In diesen, wird nur der Intensitätswert der Voxel in Betracht gezogen. Abgesehe von den niedrigen Berechnungszeiten, sind diese jedoch aus mehreren Gründen suboptimal. Medizinischen Daten werden gemessen und haben deshalb meist ein Rauschen, was die genaue Darstellung erschwert. Weiterhin sind die Intensitätswerte verschiedener Bereiche nah beieinander oder gar gleich und damit sind eindimensionale Transferfunktionen unpraktisch um verschiedene Materialien kenntlich zu machen. Trotzdem sind eindimensionale Transferfunktionen weit verbreitet und werden oft benutzt. 

\subsection{Zweidimensionale Transferfunktionen}

Eine zweidimensionale Transferfunktion hat als Eingabeparameter zwei Werte. Oftmals werden hierbei der Gradient beziehungsweise desen Länge und der Intensitätswert eine Voxels genommen.

In der Arbeit von Wesarg und Kirschner \cite{wesarg2009structure} \cite{wesarg20102d} wird das Stucture-Size-Enhanced Histogram vorgestellt. geht es um die Benutzung von 2D Histogrammen für Transferfunktionen zur Unterscheidung von verschiedenen Gewebearten, die bei CT-Bildern ähnliche Grauwerte haben.
Dabei wird für Voxel berechnet wie viele Schritt in die jeweilige Richtung der 26 Nachbarn gemacht werden kann, ohne, dass der Grauwert um mehr als eine gewisse Differenz verändert. Die Akkumulation aller Werte ergibt dann die Größe der Struktur. Hierbei wird zur weiteren Verbesserung nur der größere Wert zweier entgegengesetzter Richtungen genommen.
Das Histogramm hat folglich als zwei Eingaben die Strukturgröße und den Grauwert.




\subsection{Mehrdimensionale Transferfunktionen}




Sereda baut seine Arbeit \cite{sereda2006visualization} auf den von Serlie \cite{serlie2003computed} vorgestellten LH-Histogramen auf und zeigt wie man mit ihnen Objekte klassifizieren kann. Die Berechnung eines LH-Histograms ist eine Methode zur Erkennung von Kanten, unter der Verwendung von Low- und High-Werten. Dabei werden die Voxel in zwei verschiedenen Kategorien eingeteilt. Es gibt Voxel, die  innerhalb eines Materials liegen und welche, die  an der Grenze zweier Materialien liegen. Ist ein Voxel innerhalb, so sind seine LH-Werte gleich. Grenzvoxel hingegen haben unterschiedliche Low- und High-Werte, wobei disese die Intensitätswerte der beiden Materialien, zwischen denen die Grenze verläuft, beschreiben. 
\newline
Bei der Berechnung des Histograms wird als erstes getestet, ob der betrachtete Voxel an einer Grenze liegt. Ein Punkt liegt innerhalb eines Materials, wenn  die Länge des Gradienten kleiner als ein gewisses epsilon (bei MRT-Daten) oder gleich null(bei CT-Daten ist). In diesem Fall wären die Low- und High-Werte der Intensitätswert des Voxels. Ist dies jedoch nicht der Fall, wird in Richtung(für die High-Werte) und entgegengesetzter Richtung(für die Low-Werte) des Gradientens schrittweise integriert. Dies stoppt sobald ein Material gefunden wurde. Dies wird für jeden Punkt im Volumen berechnet und danach aus allen LH-Werten ein Histogram erstellt.
\newline 
Zur Visualisierung benutzt Sereda eine dreidimensionale Transferfunktion. Diese nimmt die beiden LH-Werte als auch die Gradientenlänge, aus dem Grund, dass vor allem Voxel nah an der Grenze interessant sind und diese dadurch hervorgehoben werden, als Parameter entgegen.
\newline
Weiterhin verwenden die Forscher Regiongrowing um Strukturen zu erkennen. Dabei basiert die Kostenfunktion auf dem LH-Histogram. Dies ist deutlich besser als Kostenfunktionen, die auf dem Intensitätswert und der Gradientenlänge basieren, da Kanten trotz Überlappungen besser erkannt werden können.
\todo{besser schreiben}
\newline
\newline
In einer späteren Arbeit \cite{sereda2006automating} stellt Sereda ein hierarschies Clusteringverfahren vor. Hierbei werden in einer Menge von Clustern immer die zwei gemerged, die sich bei dem ausgewählten Vergleichsverfahren am ähnlichsten sind. Es wird eine Kombination aus zwei solcher Vergleichsverfahren vorgestellt. Zum einen wird die räumliche Nähe in betracht gezogen, bei der gezählt wird, wie viele direkte Nachbarn zwei Cluster besitzen. Zum anderen wird die Nähe im LH-Raum untersucht. Als Startcluster dienen hierbei die Kästchen des LH-Histograms. Die einzelnen Cluster bekommen für die Visualisierung am Ende eine zufälligen Farbwert zugewiesen. 



Die Arbeit von Shouren Lan \cite{lan2017improving} befasst sich mit der Verbesserung von 2D Transferfunktionen die auf Skalarwerten und Gradienten(SG-TF) basieren. Genauer geht es darum das Problem vom Auftauchen von Überlappungen von Bereichen die nicht zusammen gehören zu vermeiden.
\newline
Dabei wird im Paper zwischen 3 verschiedenen Arten von Strukturen unterschieden:
\begin{itemize}
\item (i) Strukturen die keine andere Struktur berühren
\item (ii) Strukturen die keine andere Struktur berühren, jedoch nah an einer andern liegen
\item (iii) Strukturen die andere Strukturen berühren
\end{itemize} 
Wenn der Benutzer eine Region ausgewählt hat, werden zunächst alle Strukturen in dem Bereich klassifiziert und kleine Fragmente entfernt. Durch verschiedene Algorithmen werden Strukturen der Klassen (ii) durch Erosion, Dilatation, Aufteilen und neu zusammenfügen von einander getrennt. Strukturen der Klasse (iii) werden durch eine weitere niedrig dimensionale Transferfunktion getrennt. Anschließend werden durch das Aufteilen entstehende Löcher mit Hilfe von Dilatation gestopft. Als letztes wird durch eine Transferfunktion den Strukturen entsprechende Farben und Intensitäten zugewiesen.


\subsection{Räumliche Info}
\subsection{Machine learning}
\subsection{Clustering basierte Transferfunktionen}


Das Paper von Binh P. Nguyen \cite{nguyen2012clustering} befasst sich mit dem Anwenden von Clustering auf einem Volumenmodell um mit Hilfe einer Transferfunktion medizinische Bilder effektiv und effizient darzustellen.
In einem Vorverarbeitungsschritt wird für jeden Voxel der Gradient berechnet, und hinterher daraus die Lower und Higher Intensitätswerte. Anschließend wird aus diesen Daten ein LH Histogramm erstellt.
In den ersten beiden Clusteringschritten werden zunächst Voxel mit ähnlichen  LH-Werten in Clustern zusammengefasst. Anschließend werden diese Cluster erneut in mehrere Cluster aufgeteilt, bei denen die Voxel auch im Raum nah beieinander liegen. Für diese beiden Schritte wird jeweils ein Parameter benötigt, der die maximale Distanz zum mittleren Voxel in einem Cluster bestimmt. Daraufhin werden alle Cluster hierarchisch gemerged, bis es nur noch einen gibt. Dabei werden immer die zwei paarweise räumlich am nächsten ausgesucht und es wird gespeichert welche Cluster gemerged wurden. Anschließend kann der Benutzer durch umkehren des Algorithmus entscheiden wie viel Cluster er haben möchte. Am Ende werden die Cluster mit verschieden Farben und Intensitätswerten versehen.

\subsection{Imagebased}
Eine weitere Art Transferfunktionen anzuwenden, sind Verfahren, die auf ?einem Bild? basieren. Hier hat der Benutzer die Möglichkeit, mit dem Programm zu interagieren. Dies ist für einen unerfahrenen User intuitiver und er kann durch ausprobieren ein gewünschtes Ergebnis erzielen.

Fang stellt in seinem Paper \cite{fang1998image} ein Verfahren vor, bei dem die  Transferfunktioneine eine Abfolge von verschieden 3D Bildverabteitungsverfahren ist, deren Parameter vom Benutzer angepasst werden können.

Es gibt auch Methoden, bei denen ein Hilfswerkzeug zum Einsatz kommt. Zum Beispiel kann der Benutzer im Paper von Reitinger: \cite{reitinger2004user} sich gezielt Bereiche des Volumens hervorheben lassen, indem er sie mit einem Stift auswählt. Hierbei wird die Intensiät des ausgewählten Punktes als auch der räumliche Abstand zum Stift in Betracht gezogen.



%% ==============================
\chapter{\iflanguage{ngerman}{Methoden}{Methods}}
\label{sec:methods}
%% ==============================


Diese Arbeit orientiert sich an dem zweistufigen Clusteringverfahren aus dem Paper von Nguyen \cite{nguyen2012clustering}. 

\subsection{Gradient}

Zuerst müssen die LH-Werte des Volumen berechnet werden. Hierfür werden zunächst die Gradienten aller Voxel benötigt. Wie im Paper beschrieben wurde auch in dieser Arbeit Hong's Methode \cite{hong2003method} dafür gewählt.
Diese ist ein approximationsbasiertes Verfahren zur Berechnung von Gradienten eines Volumens. 
\newline
In Hong's Verfahren wird zur Berechnung der Gradienten die lokale 4x4x4 Nachbarschaft des betrachteten Punktes hinzugezogen. Hierbei ist zu beachten, dass es nicht möglich ist, den Gradienten für einen Voxel direkt zu berechnen. Der Gradient drückt die Veränderung der Intensitätswerte im Raum aus, folglich kann er immer nur zwischen zwei Punkten berechnet werden. Deshalb liegt er im Falle eines dreidimensionalen Volumens im Zentrum eins Würfels, der von 8 benachbarten Voxeln aufgespannt wird. In \autoref{fig:nachbarschaft} ist zu erkennen, wie der Gradient im Zentrum der acht Voxel liegt. Desweiteren ist die 4x4x4 Nachbarschaft in Form der durchnummerierten Punkte zu sehen.
\newline

\begin{figure}[!h] 
\centering 
\includegraphics[width=\textwidth]{Logos/VoxelEdges.PNG}
\caption{Darstellung der lokalen 4x4x4 Nachbarschaft} 
\label{fig:nachbarschaft} 
\end{figure}
\todo{richtig bild zitieren u. evtl kleiner}

Die Funktionen für die Intensitätswerte wird im Paper mit:
\begin{equation}
	f(x,y,z) = Ax^{2}+By^{2}+Cz^{2}+2Fyz+2Gzx+2Hxy+2Ix+2Jy+2Kz+D
\end{equation}
approximiert. Da der Gradient die Ableitung der Intensitätsfunktion ist, erhält man den dreidimensionalen Gradientenvektor $n$, indem die Funktion ableitetet wird:
\begin{equation}
	n = (Ax+Gz+Hy+I, By+Fz+Hx+J, Cz + Fy + Gx + K)
\end{equation}
Um den Gradienten zu Berechnen müssen die Parameter $A,B,C,E,F,G,H,I,J,K$  berechnet werden. Dies geschieht mithilfe der Methode der kleinsten Quadrate.
\todo{methode der kleinsten quadrate genauer beschreiben}

\subsection{LH-Werte}

Als nächster Schritt, nach der Kalkulation der Gradienten, folgt die Berechnung der Low- und High-Werte. Dazu wird in Richtung der Gradienten integriert. Hierfür wurde, wie auch von Nguyen, Heun's Methode, eine modifizierte Euler Methode, verwendet. Die für die Integration benutzte Formel lautet:
\begin{equation}
	u_{i+1} = u_{i} + \frac{1}{2}d(\triangledown f (u_{i}) + \triangledown f(u_{i}+d \triangledown f(u_{i}))) 
\end{equation}
Hierbei sind $u_{i}$ und $u_{i+1}$ die Positionen des aktuellen, beziehungsweise des nächsten Voxels. $\triangledown f(x)$ beschreibt den normalisierten Gradienten für die High-Werte und den normalisierten inversen Gradienten für die Low-Werte an Stelle $x$ . $d$ steht für die Schrittweite, die in dieser Arbeit auf einen Voxel festgelegt wurde.
Die Integration stoppt, wenn eine lokale Extremstelle oder ein Wendepunkt erreicht wird. Dies ist daran zu erkennen, dass die Länge des Gradienten an dieser Stelle null ist. Bei MRT-Daten muss ein Grenzwert festgelegt werden, da Gradienten nie null werden. Bei CT-Daten ist dies jedoch kein Problem. Wird ein solcher Punkt erreicht, wird der Intensitätswert dieses Voxels als Ergebnis für den Low- beziehungsweise High-Wert des Startvoxel gespeichert.
\newline
Anschließend wird ein LH-Histogramm mit allen berechneten LH-Wertpaaren erstellt. Hierbei sind auf der x-Achse die Low-Werte und auf der y-Achse die High-Werte angesiedelt. Die Werte der Achsen reichen von null bis zum jeweiligen Maximum des Low- beziehungsweise High-Werte.

\subsection{LH-Clustering}

Als nächstes, wird der erste Clusteringschritt berechnet. Dieser findet im LH-Raum statt, genauer wird er auf dem eben berechneten LH-Histogramm angewendet. Dabei kommt Meanshiftclustering zum Einsatz. Der Ablauf davon geschieht wie folgt.
Vor dem Clustering muss eine Bandweite und ein Thresholdwert bestimmt werden, welche die Sensitivität des Clusterings festlegen. Die Bandweite liegt im Paper von Nguyen \cite{nguyen2012clustering} bei 7\% - 9\% des maximalen LH-Wertes und der Threshold bei 0,01. Anschließend kann das LH-Clustering auf jeden Punkt im Histogramm angewandt werden.
\newline
Für einen beliebigen Punkt, werden alle Punkte die innerhalb des Radius der Bandbreite liegen gespeichert. Diese Punkte bilden nun den gefundenen Cluster. Von diesem Cluster wird der neue Mittelpunkt, der jeweilige Mittelwert der beiden Koordinaten, berechnet. Um diesen Punkt wird erneut mit selben Radius alle Punkte die bisher nicht zu dem Cluster gehören gesucht und hinzugefügt. Dies geschieht solange, bis der Abstand des neu kalkulierte Mittelpunkt zum alten weniger als der Thresholdwert multipliziert mit die Bandweite ist. 
Nachdem dieses Prozedere für jeden Punkt im Histogramm ausgeführt wird, gibt es viele verschiedene Cluster. In einem nächsten Schritt, werden alle Cluster die sehr nah beieinander liegen verschmolzen. Dies betrifft jene Cluster, deren Mittelpunkte eine Distanz kleiner als die Hälfte der Bandweite zueinander haben.

\subsection{Räumliches-Clustering}

Als letzten Schritt, wird erneut auf jedem eben entstandenen Cluster Meanshiftclustering angewendet. Dabei wird auf jedem Cluster einzeln und unabhängig von den anderen Clustern geclustert. Außerdem werden diesmal die räumlichen Informationen der Punkte in Betracht gezogen. Hierzu müssen zunächst erneut die beiden Parameter Bandweite und Threshold definiert werden. Das Clustering läuft wie zuvor im LH-Raum ab, nur wird statt im zweidimensionalem Raum mit einem zweidimensionalem Kreis im dreidimensionalem Raum des Volumens mit einer dreidimensionalen Kugel geclustert. Des Weiteren findet am Ende der Berechnung der Cluster keine Verschmelzung statt, da dies den Sinn der beiden verschiedenen Clusteringschritte zerstören würde. In den finalen Cluster haben alle Punkte jeweils ähnliche LH-Werte und liegen im Volumen nah beieinander. Würde man diese Cluster anhand ihrer räumlichen Informationen verschmelzen, haben die Punkte keine ähnlichen LH-Werte mehr und der erste Clusteringschritt wäre umsonst gewesen.
\todo{vllt mehr bei räumlich clustering}





Der vorgestellte dritte hierarchische Clusteringsschritt wurde in dieser Arbeit nicht angewendet. Dies geschah aus dem Grund, dass ...
\todo{warum nicht hierarchisch}


%% ==============================
\chapter{\iflanguage{ngerman}{Design}{Concept}}
\label{sec:concept}
%% ==============================


\todo{Formulierung}
Nachdem im vorherigen Kapitel die Methode der Implementierung vorgestellt wurde, beschäftigt sich dieses mit dem Softwaredesign der Implementierung dieser.
\newline
Die Implementierung dieser Bachelorarbeit teilt sich dabei in zwei verschieden Programme auf. Zum einen den sogenannten \textit{VolumeRenderHelper}, der für das Laden, Umwandeln, Verarbeiten und Speichern der Volumendaten zuständig ist. Zum anderen das Unityprogramm \textit{VolumeRenderer}, welches zur Visualisierung der vom Helper erzeugten Daten dient. Diese beiden Programme existierten bereits vor dieser Arbeit und sind in Vorarbeiten des IPRs entstanden. In dieser Bachelorarbeit wurde der \textit{VolumeRenderHelper} um die vorgestellte Transferfunktion erweitert und der \textit{VolumeRenderer} geringfügig angepasst. Die Beiden Programme werden im folgenden als Helper und Renderer erwähnt. Weiterhin wurde ein Pythonskript \textit{PlotHelper} geschrieben, um LH-Histogramme anzuzeigen.
\newline
\todo{christians paper}
\todo{unity und stefffens zeug genauer erklären}
\todo{csharp erwähnen und allgemeine umgebung}
Anfangs liegen die CT-Daten von den Volumen in mehreren Dateien als Schnittbilder im DICOM Format vor. Diese werden mithilfe der MITK Workbench zu einer einzelnen Datei im .nrrd Format umgewandelt, da nur Volumen in diesem Format vom Helper eingelesen werden können.
\newline
Die interne Speicherung des Volumens wird mit der generischen Klasse \textit{Volume} umgesetzt. Diese besitzt ein dreidimensionalen Array vom angegebenen generischen Datentyp, sowie Informationen über das Volumen, wie zum Beispiel Größe der Voxel oder Anzahl der Elemente pro Achse. Des Weiteren bietet die Klasse verschieden Funktionen um Informationen auszulesen oder zu bearbeiten. Zur Darstellung von dreidimensionale Koordinaten, werden die Klassen \textit{IVector3} und \textit{FVector3} benutzt. Diese stellen Vektoren dar, die entweder Integer, Ganzzahlen, oder Float, Dezimalzahlen, als $x,y$ und $z$ Werte speichern.
\todo{Volumenklasse  UML}
\newline
\todo{mergecluster oder clustermerge}
Die Interaktion des Benutzers mit dem Helper findet über eine Kommandozeile statt. Hierbei hat der Anwender die Befehle \textit{Load}, \textit{Dump}, \textit{Resample}, \textit{Info}, \textit{Write}, \textit{LHHistogram}, \textit{ClusterVolume} und \textit{MergeCluster} zur Auswahl. Jedes Modul hat dabei seine eigene Syntax die mithilfe eines Help Befehls angezeigt werden kann.
\newline
Die Funktionen der Module entsprechen deren Namen. So lädt \textit{Load} bespielsweise eine .nrrd oder binäre Datei, \textit{Dump} und \textit{Write} speichern das geladene Volumen als binäre oder .nrrd Datei ab, \textit{Info} gibt Informationen über das aktuell geladene Volumen zurück und \textit{Resample} lässt den Anwender die Größe des Volumens verändern. Ruft man den \textit{Dump} Befehl mit einem $u$ am Ende auf, so wird das Volumen vor dem Speichern zum Typ $unsigned$ $int$ gecastet. Dies geschieht indem auf alle Werte der Betrag des minimalen Wertes aufaddiert wird. Dadurch verschieben sich die Werte so, dass das Minimum bei null liegt, also nur noch positive Zahlen im Volumen vorhanden sind. Dies ist für die Darstellung im Renderer wichtig, da dieser nur mit positiven Zahlen funktioniert.
\newline
Im Folgenden werden hauptsächlich die Module \textit{LHHistogram}, \textit{ClusterVolume} und \textit{MergeCluster} erläutert, da diese im Laufe dieser Arbeit entstanden sind. \textit{LHHistogram} berechnet das LH-Histogramm des geladenen Volumens und speichert dieses in einer .csv Datei ab. \textit{ClusterVolume} kalkuliert ebenfalls die LH-Werte des Volumens, führt jedoch hinterher noch die beiden Clusteringsschritte des Verfahrens aus. Als Ausgabe speichert das Modul eine binäre Datei eines Volumens, in welchem die verschiedenen IDs der Cluster gespeichert sind. Die Idee dahinter wird im laufe diese Kapitels erklärt. Das Modul \textit{MergeCluster} dient der Verschmelzung der gewünschten IDs mit dem ursprünglichen Volumen. Das Ergebnis dessen, ist eine finale binäre Datei, die im Renderer das Ventrikelsystem visualisiert. Ein Überblick über den gesamten Aufbau und Ablauf der Implementierung ist in  \autoref{fig:ueberblick} zu sehen.

\begin{figure}
\centering 
\includegraphics[width=\textwidth]{Logos/Ueberblick.png}
\caption{Überblick über das Programm} 
\label{fig:ueberblick} 
\end{figure}

\todo{abbildung die zeigt warum es das volumen kleiner wird}
\todo{funktionsnamen}
Der Ablauf der Berechnung startet in der statischen \textit{Gradient} Klasse. Diese ist eine Implementierung des Verfahren von Hong \cite{hong2003method}. Zur Berechnung wird der Funktion \textit{bla} das Volumen der Intensitätswerte, als Volumen aus Integern, als Parameter übergeben. Wie im vorherigen Kapitel besprochen, können die Gradienten nicht für einen Voxel direkt berechnet werden, sondern nur für die Punkte zwischen den Voxeln. Aus diesem Grund ist das Ergebnisvolumen um ein Voxel in jeder Achse kleiner als das Volumen der Intensitätswerte, da... (abbildung). Als Ergebnis der Funktion wird ein Volumen vom Typen \textit{FVector3} zurückgegeben.
\newline
DIe Berechnung der LH-Werte findet in der statischen Klasse \textit{LHValues} in der Funktion \textit{LHValueVolume} statt. Als Parameter wird das \textit{FVector3} Volumen der Gradienten aus dem Schritt davor entgegengenommen. Da für die Berechnung der LH-Werte die Intensitätswerte und die Gradienten am gleichen Punkt vorhanden seien müssen, müssen die Intensitätswerte für das verschobene Volumen der Gradienten berechnet werden. Dazu wurde eine einfache Interpolation durchgeführt, indem von allen 8 Nachbarn eines Punktes die Intensitätswerte aufaddiert und hinterher durch acht geteilt wurden. Hierbei muss jedoch beachtet werden, dass die Werte dadurch verändert werden, und Informationen verloren gehen.
\newline
Hat der Benutzer das Modul \textit{LHHistogram} aufgerufen, wird im Anschluss das LH-Histogramm in der Klasse \textit{LHHistogramCSV} erstellt und wird von ihr als .csv Datei in einem vom Anwender angegebenen Pfad abgespeichert.
\newline
An dieser Stelle kommt das Pythonskript \textit{PlotHelper} zum Einsatz. Dieses lädt die .csv Datei und visualisiert das LH-Histogramm mithilfe einer kalt-zu-heiß-Farbrampe in einem zweidimensionalem Koordinatensystem.
\newline
Hierbei ist zu beachten, dass das Histogramm abhängig von der Häufigkeit des Vorkommens eines LH-Wertpaares im Volumen gebildet wird. Diese werden im jeweils dazu passenden Kästchen des Koordinatensystems gespeichert. Dies ist ein simplerer Vorgehen, als das im Paper von Nguyen \cite{nguyen2012clustering} benutzte Erstellen des Histogramms abhängig von einer für jeden Voxel berechneten Gewichtung. Da die Arbeit an der Implementierung zeitlich beschränkt war, wurde diese Gewichtung, die einzig und allein einer genaueren Darstellung des für das Verfahren irrelevante LH-Histogramm dient, vernachlässigt. Die Gewichtung ist für das Clustering belanglos, da dort ein Histogramm wie oben beschrieben, abhängig von der Häufigkeit der LH-Werte verwendet wird. 
\newline
Wurde jedoch das \textit{ClusterVolume} Modul aufgerufen, wird mit den beiden Clusteringschritten fortgefahren.
Die Berechnung die in der \textit{LHClustering} Klasse geschieht nimmt das Volumen mit den LH-Werten entgegen und rechnet dieses aus Performancegründen, wie oben beschrieben, in ein ein Histogramm um. Dieser Schritt könnte gespart werden, wenn die Methode \textit{LHValueVolume} direkt ein Histogramm als Rückgabewert liefern würde. Dies würde auch das \textit{LHHistogram} Modul verbessern, da damit die Umrechnung in dieser Klasse ebenso hinfällig wird. Als Ergebnis der Clusteringfunktion \textit{ComputeLHClusters} wird eine Liste der Cluster zurückgegeben, wobei ein Cluster aus einer Liste von \textit{IVector3} besteht. Die Cluster werden nur als Liste der räumliche Informationen der Voxel gespeichert, da für den nächsten Clusteringschritt  lediglich diese Information benötigt wird. 
\newline
Anschließend geht es in der \textit{SpatialClustering} Klasse mit der Berechnung der räumlichen Cluster weiter...
\todo{mehr räumlich clustern beschreiben}
\newline
Nachdem alle Cluster kalkuliert wurden, werden sie in einem Volumen gespeichert. Jeder Cluster bekommt dabei zunächst seine eigene ID beginnend bei eins. Das Volumen wird dann mit den verschiedenen IDs gefüllt. Dies geschieht indem für jeden Cluster an den Positionen der Punkte die jeweilige ID gespeichert wird. Alle anderen Voxel des Clustervolumen wird der Wert null zugewiesen. Dieses Volumen wird als binäre Datei gespeichert und ist das Ergebnis der ClusterVolume Moduls. Es ist bei diesem Modul sehr wichtig, dass es, wie es bei dem Dump Befehl möglich ist, mit einem $u$ am Ende aufgerufen wird, da das Ergebnis sonst nicht vom Renderer dargestellt werden kann.
\newline
Das Ergebnis muss anschließend vom Nutzer in Unity geladen werden. Hierbei kommt die von dieser Arbeit vorgenommenen Anpassung am Renderer zum Einsatz. Dadurch ist es dem Benutzer möglich über ein Eingabefeld während der Visualisierung einen Wert, oder einen Wertebereich anzugeben. Dieser wird dann in einer gewählten Farbe, standardmäßig rot, hervorgehoben. Dies muss der Anwender nutzen, um die IDs, die das Ventrikelsystem beschreiben, zu finden.
\newline
Hat er dies getan kann er mit dem\textit{MergeCluster} Modul des Helpers das Ergebnis zusammenfügen und die finale Visualisierung erhalten. Beim Aufruf muss der Nutzer das Intensitätsvolumen, das Clustervolumen mit den IDs und die ausgewählten IDs als Parameter übergeben. Anschließend wird an den Stellen der ClusterIDs die Werte im Intensitäsvolumen mit dem Wert 5000 überschrieben, da nach dem beschriebenen Werteshift ins positive der maximale Intensitätswert bei ungefähr 4500 liegt. Dies erhöht das Maximum des Volumens nur gering und lässt es zu die zu visualisierenden Bereiche klar vom Rest abzugrenzen. Dieses Volumen wird erneut als binäre Datei gespeichert.
\newline
Als letzten Schritt kann nun der Benutzer den Wert der eben erklärte Erweiterung des Renderer auf 5000 stellen und die Visualisierung des Ventrikelsystems betrachten.














































%% ==============================
\chapter{\iflanguage{ngerman}{Implementierung}{Implementation}}
\label{sec:implementation}
%% ==============================




Nachdem im letzten Kapitel das Softwaredesign besprochen wurde, beschäftigt sich das Kommende mit der Implementierung an sich.


Wie im Kapitel \nameref{sec:methods} schon erwähnt, wurden bei der Implementierung verschiedene Ausnahmen getroffen, um das Ergebnis und die Rechenzeit zu verbessern. Beispielsweise 


Bei der Berechnung der LH-Werte passiert es, bei ungefähr 2-\%-3\% der Fälle, dass ein Iterationsschritt außerhalb des Volumens landen würde. In diesem Fall, wird der letzte besuchte Voxel als Ergebnis für die Integration genommen. Des Weiteren kommt es bei zirka 25\% aller Berechnung dazu, dass die LH-Werte vertauscht waren, also der Low- größer als der High-Wert war. Dem wird entgegengewirkt, indem bei einem Vorkommen dieses Problems die beiden Werte vertauscht gespeichert werden. Es war noch nicht möglich den Grund für diese Verwechslung herauszufinden.

Es ergaben sich bei der Implementierung verschiedene Probleme. Beispielsweise wurde anfangs das Verfahren mit MRT-Daten getestet. Dies war von wenig Erfolg, da die Volumendaten große Unterschiede zu den CT-Daten aufweisen und somit das Verfahren nicht funktionierte.



Bei der Kalkulation der Gradienten, wird eine Gewichtung und die Koordinaten aller 64 Punkte in der lokalen Nachbarschaft benötigt. Da das gesamte Volumen die selbe Voxellänge hat und die Koordinaten in der lokalen Nachbarschaft immer gleich sind, konnten diese beiden Werte für alle 64 Nachbarn einmalig in einem Vorverarbeitungsschritt berechnet werden. Sie werden dabei in einem 64 Elemente großes Array mit der gleichen Nummerierung wie in \autoref{fig:nachbarschaft} gespeichert. Die Implementierung ist in \autoref{fig:gewicht_koord} zu sehen. Es muss bei Kalkulation jedes Gradienten lediglich die beiden Arrays durch iteriert  werden.


\begin{figure}[!h] 

\includegraphics[width=1.2\textwidth]{Logos/Nachbar_Code_Hell_Kurz.PNG}
\caption{Implementierung der Berechnung der Gewichte und Koordinaten} 
\label{fig:gewicht_koord} 
\end{figure}



\begin{figure}[!h] 
\includegraphics[width=1.2\textwidth]{Logos/LH_Code.PNG}
\caption{Implementierung der Berechnung der High-Werte} 
\label{fig:lh_code} 
\end{figure}


In \autoref{fig:lh_code} kann man den Code zur Berechnung eines High-Wertes sehen. Der Code liegt innerhalb einer while-Schleife, die so lange aufgerufen wird, bis \textit{HisFinished} und \textit{LisFinished} true sind. Die Berechnung des Low-Wertes ist ebenfalls in der Schleife und sieht bis auf die Richtung der neuen \textit{absVector} und \textit{secondAbsVector} gleich aus.
\newline
Am Anfang eines Durchlaufes wird der normalisierte Gradienten des aktuellen Punktes in \textit{absVectorNormalized} gespeichert. Anschließend wird der Punkt des zweiten normalisierten Vektors berechnet. Liegt dieser außerhalb des Volumens, so wird die Integration beendet. und der Wert des letzten besuchten Voxels als Ergebnis zurückgegeben, da das Ergebnis immer der letzte Eintrag der Liste \textit{highestIntensity} ist. Liegt er jedoch innerhalb des Volumens, so wird der \textit{secondAbsVector} berechnet und der Intensitätswert des aktuellen Punktes zu der Liste hinzugefügt. Ist der Gradient des aktuellen Punktes kleiner als \textit{extremumLength}, was im Falle von CT-Daten bei null liegt, wird die Integration beendet. Andernfalls wird der neue Punkt mithilfe von \textit{absVectorNormalized} und \textit{secondAbsVecotr} gemäß Hong's Verfahren \cite{} ermittelt. Erneut endet die Integration, falls dieser Punkt außerhalb des Volumens liegt. Vor der Berechnung des nächsten Punktes, wird jedoch über die \textit{highestIntensity} Liste nach Schleifen gesucht. Bei sehr kleinen Gradienten, die jedoch größer als null sind, kann es vorkommen, dass das Verfahren immer wieder den selben Punkt findet und somit in einer Endlosschleife feststeckt. Diese Kontrolle könnte jedoch über die Koordinaten der iterierten Punkte geschehen, um dem unwahrscheinlichem Fall, dass zwei durch die Integration hintereinander gefundenen Punkte genau den selben Intensitätswert haben.
\newline
Dieser Vorgang wiederholt sich für den High- als auch für den Low-Wert solange, bis die Integration aus einem der gegebenen Abbruchkriterien stoppt.
%% ==============================
\chapter{\iflanguage{ngerman}{Ergebnisse}{Results}}
\label{sec:results}
%% ==============================





Diese Kapitel beschäftigt sich mit den Ergebnissen und der Evaluation dieser Bachelorarbeit. Die Ergebnisse der Visualisierung sind in mehreren Abbildungen zu sehen. Die Evaluation teilt sich folgendermaßen auf. Als erstes werden die Ergebnisse der Visualisierung des Ventrikelsystems gezeigt und evaluiert. Dabei wurde für die Auswertung ein Interview mit einem Arzt durchgeführt. Danach wird eine Nutzerstudie vorgestellt, die die Benutzerfreundlichkeit des Systems testet. Als letztes wird die Berechnungszeit der Implementierung besprochen.


\subsection{Visualisierung}

Die in diesem Unterkapitel gezeigten Visualisierungen wurde alle auf Volumen mit einer Auflösung von 256x101x256 Pixeln berechnet. Aus dem Grund, dass dies ein gute Balance zwischen der Auflösung und der Berechnungszeit der Ergebnisse darstellt. Im Unterkapitel  Berechnungszeiten wird dies genauer beschrieben. Bevor die Visualisierung ausgewertet werden kann, wird zunächst das Ventrikelsystem erläutert.
\newline
In \autoref{fig:ventrik} ist das Ventrikelsystem zu sehen. Dieses besteht aus vier verschiedenen Ventrikeln. Zum einen der linke und der rechte Seitenventrikel, oberen beiden Bögen, die in der Abbildung zu sehen sind. Zum anderen der dritte und vierte Ventrikel, die zwischen den beiden Seitenventrikeln liegen und nach unten weggehen. Die Seitenventrikel bestehen aus einem Vorderhorn, Nr. 1,  einem Hinterhorn, Nr. 2, und einem Unterhorn, Nr. 3. Der dritte Ventrikel ist mit der Nummer 4 und der vierte Ventrikel mit der Nummer 5 versehen.
\newline
Bei der Ventrikelpunktion, wird einer der beiden Seitenventrikel im vorderen Bereich punktiert. Dies macht vor allem die Darstellung der Seitenventrikel relevant.

\begin{figure}[!h] 
\centering 
\includegraphics[width=0.75\textwidth]{Logos/Ventrikelsystem_V2.png}
\caption{Zeichnung des Ventrikelsystems  \\  Quelle:$ http://www.kiefer-saarland.com/anatomie\_physiologie.htm$} 
\label{fig:ventrik} 
\end{figure}


Das Verfahren wurde an 15 verschiedenen CT-Datensätzen von der Uni Ulm getestet. Darunter waren verschiedene Ventrikelsysteme. Vier der Datensätze waren von Menschen mit einem normalen Ventrikelsystem, vier andere wiesen ein sehr schlankes System auf. Weiterhin litten zwei Patienten unter Atrophie, einem, oft durch das Alter verursachtem, Schwund an Gehirnmasse. Zwei andere Datensätze waren von Leuten, die unter einem Mittellinienshift litten. Dies bezeichnet die Verschiebung der Mittellinie der Ventrikelsystems, was beispielsweise durch einen Schlag auf den Kopf hervorgerufen werden kann. Weiterhin gab es Daten, von einer Hirnblutung, und einem Hydrocephalus, einer Aufstauung von Nervenwasser im Kopf. Als letztes gab es noch Daten eines deformierten Ventrikelsystems, ausgelöst durch Wassereinlagerungen im Gehirn.
\newline
Eine erfolgreiche Visualisierung gelang in nur 3 der 15 Fälle und zwar bei zwei Datensätzen mit normalen Ventrikeln und einem von einem Patient, der unter Atrophie leidet. In \autoref{fig:norm1_s} und \autoref{fig:norm1_u} ist die Visualisierung des ersten normalen Ventrikelsystems, in \autoref{fig:norm2_s} und \autoref{fig:norm2_u} die Visualisierung des zweiten normalen Ventrikelsystems und schließlich in \autoref{fig:atro_s} und \autoref{fig:atro_u} die Visualisierung des Ventrikelsystems des Patienten mit Atrophie zu sehen. Die Abbildungen zeigen Screenshots aus der Darstellung in Unity, jeweils aus den Perspektiven von der Seite und von oben.

\begin{minipage}[c]{0.49\textwidth}
\includegraphics[width=\textwidth]{Logos/Normal1/Seite.PNG}
\captionof{figure}{Visualisierung des ersten normalen Ventrikelsystems von der Seite}
\label{fig:norm1_s}
\end{minipage}
\begin{minipage}[c]{0.49\textwidth}
\includegraphics[width=\textwidth]{Logos/Normal1/Unten.PNG}
\captionof{figure}{Visualisierung des ersten normalen Ventrikelsystems von Unten}
\label{fig:norm1_u}
\end{minipage}


In \autoref{fig:norm1_s}, der Darstellung des ersten normalen Ventrikelsystems von der Seite, sind links über dem Hinterhorn der Seitenventrikel Punkte zu erkennen, die nicht zum Ventrikelsystem gehören. Auch bei den Visualisierungen der andern Ventrikelsysteme sind an mehreren Stellen solche Ausreißer zu erkennen. Diese können bei dem aktuellen Stand der Implementierung nicht entfernt werden und senken die Qualität der Darstellung. 



\begin{minipage}[c]{0.49\textwidth}
\includegraphics[width=\textwidth]{Logos/Normal2/Seite.PNG}
\captionof{figure}{Visualisierung des zweiten normalen Ventrikelsystems von der Seite}
\label{fig:norm2_s}
\end{minipage}
\begin{minipage}[c]{0.49\textwidth}
\includegraphics[width=\textwidth]{Logos/Normal2/Unten.PNG}
\captionof{figure}{Visualisierung des zweiten normalen Ventrikelsystems von Unten}
\label{fig:norm2_u}
\end{minipage}



\begin{minipage}[c]{0.49\textwidth}
\includegraphics[width=\textwidth]{Logos/Atrophie/Seite.PNG}
\captionof{figure}{Visualisierung des Atrophie Ventrikelsystems von der Seite}
\label{fig:atro_s}
\end{minipage}
\begin{minipage}[c]{0.49\textwidth}
\includegraphics[width=\textwidth]{Logos/Atrophie/Unten.PNG}
\captionof{figure}{Visualisierung des Atrophie Ventrikelsystems von Unten}
\label{fig:atro_u}
\end{minipage}




Für die drei erfolgreich visualisierten Daten wurde zur Evaluation ein Arzt von der Uniklinik Ulm interviewt. Diesem wurden die Visualisierungen direkt in Unity vorgeführt. Während der Vorführung konnte der Mediziner selbst die Kamera durch die Darstellung lenken und das Ergebnis aus verschiedenen Blickwinkeln betrachten. Anschließend bewertete er die Qualität der Ergebnisse, indem er drei verschiedenen Fragen zu den Darstellungen beantwortete. Er musste dabei seine Antwort immer auf einer Skala von 1 bis 5 angeben.
\newline
Die Fragen lauteten:
\begin{itemize}
	\item 1) Wie gut ist das Ventrikelsystem bei der Visualisierung zu erkennen? \newline sehr schlecht 1 - 5 sehr gut
	\item 2) Wird das Ventrikelsystem in der Visualisierung vollständig dargestellt? \newline überhaupt nicht vollständig 1 - 5 vollständig
	\item 3) Wie genau ist das Ventrikelsystem segmentiert? \newline überhaupt nicht segmentiert 1 - 5 ausschließlich das Ventrikelsystem ist segmentiert
\end{itemize}


Die Ergebnisse der Antworten des Arztes werden in \autoref{tab:ergebnis_arzt} gezeigt. Allgemein merkte der Mediziner jedoch an, dass bei jeder Visualisierung, die durch das Verfahren erzeugt wurde, nur der linke und rechte Seitenventrikel zu sehen war. Die deutlich schmaleren dritten und vierten Ventrikel und die Unterhörner der Seitenventrikel fehlten bei den Darstellungen komplett. Er erklärte jedoch, dass diese bei einem gesunden Menschen jedoch sehr dünn und deshalb anhand von CT-Daten schwer zu segmentieren seien. Weiterhin sind, wie schon erwähnt, die Seitenventrikel entscheidend für eine erfolgreiche Punktion. In folge dessen, wurde sich darauf geeinigt, dass die Beantwortung der zweiten Frage, nach der Vollständigkeit des Ventrikelsystems, lediglich auf die Vollständigkeit der beiden Seitenventrikel bezogen ist.

\begin{table}[h]
\centering
\resizebox{\columnwidth}{!}{
 \begin{tabular}{| c | c | c | c |}
  \hline
  Ventrikelsystem & 1. Frage & 2.Frage & 3. Frage\\ \hline
  Normal 1 & 4 & 4 & 3 \\ \hline
  Normal 2 & 4 & 2 & 3\\ \hline
  Atrophie & 4 & 3 & 2 \\ \hline
 \end{tabular}
 }
\caption{Ergebnisse des Interviews mit einem Arzt}
\label{tab:ergebnis_arzt}
\end{table}


Die Bewertung des ersten normalen Ventrikelsystems fiel positiv mit 11 von möglichen 15 Punkten aus. Das Ventrikelsystem war als solches klar und deutlich zu erkennen, jedoch fehlte ein Teil der Hinterhörner. Bis auf diese waren die Seitenventrikel jedoch vollständig zu sehen. Weiterhin war die Segmentierung nicht exakt, da es mehrere kleine Ausreißer gab.
\newline
Das zweite normale Ventrikelsystem, war ebenfalls klar zu erkennen. Jedoch wurde fast ausschließlich der linke Seitenventrikel segmentiert. Die Bewertung der Vollständigkeit fiel mit einer vier so hoch aus, da der eine Seitenventrikel sehr klar, deutlich und vollständig zu sehen war. Der Arzt sagte, dass dieser besser und glatter als die des ersten normalen Ventrikelsystems sein, da auch das Hinterhorn zu erkennen ist. 
\newline
In einem Gehirn eines unter Atrophie leidenden Menschen, ist deutlich mehr Liquor, als bei einem gesunden Menschen, zu finden. Diese Flüssigkeiten werden vom Verfahren ebenfalls erkannt, weshalb die Segmentierung etwas verschwommen erscheint und nicht die kompletten Seitenventrikel erfasst werden. Jedoch vermutete der Arzt, dass viele der Ausreißer zum dritten Ventrikel gehören, da sich dieses, wie alle Ventrikel bei einer Atrophie, weitet. Trotz der Flüssigkeiten im Gehirn war das Ventrikelsystem in der Darstellung deutlich zu erkennen.
\newline
 Im Allgemeinen sagte der Mediziner, dass das Ventrikelsystem bei allen Visualisierungen als solches eindeutig zu erkennen sei. Er bemängelte jedoch, dass die Darstellungen nicht glatt genug sei und zu viele Ausreißer besitzen.


Zur weiteren Auswertung der Ergebnisse wurde das erste normale Ventrikelsystem aus \autoref{fig:norm1_s} und \autoref{fig:norm1_u} in Unity und mit der MITK-Workbench, die auch zum Umwandeln der DICOM-Dateien benutzt wurde, visualisiert.
\newline
In Unity wurde dabei die Erweiterung zum hervorheben von Wertebereichen benutzt. Diese wurde auf 1025 bis 1030 eingestellt, da das Ventrikelsystem Intensitätswerte in diesem Bereich hat. Die Ergebnisse sind in \autoref{fig:unity_s} und \autoref{fig:unity_u} zu sehen. Dieses Vorgehen entspricht einer simplen eindimensionalen Transferfunktion, die abhängig von den Intensitätswerten Voxel einfärbt.
\newline
Das Ventrikelsystem ist zwar zu sehen, es gibt jedoch sehr viele Ausreißer, die es fast unmöglich machen die Ventrikel zu erkennen. Diese simple Vorgehensweise führt zu keinem gewünschtem Ergebnis.
\newline
In der MITK-Workbench gibt es verschiedene Segmentierungstools. Darunter ist ein Region Growing Tool, bei dem der Nutzer einen \textit{seed} Punkt in den Schnittbildern wählen kann. Über die verwendete Kostenfunktion werden keine Informationen genannt. Anschließend können die Löcher der Auswahl mit einem \textit{closing} Filter geschlossen und mit einem weiteren Tool ein geglättete 3D Ansicht des Ergebnis erzeugt werden. Screenshots des Ergebnisses sind in \autoref{fig:mitk_o} und \autoref{fig:mitk_v} zu sehen.
\newline
In dem von der Workbench erzeugte Ergebnis ist das Ventrikelsystem gut zu erkennen. Die Seitenventrikel sind vollständig und besitzen eine glatte Oberfläche. Sogar große Teile des dritten und vierten Ventrikels sind teil der Segmentierung. Allerdings sind auch ganze Bereiche hervorgehoben, die nicht zum Ventrikelsystem gehören. Diese könnten vom Benutzer manuell in jedem Schichtbild des Datensatzes entfernt werden, was jedoch sehr zeitaufwändig wäre.


\begin{minipage}[c]{0.49\textwidth}
\includegraphics[width=\textwidth]{Logos/Normal1_Unity/Seite.PNG}
\captionof{figure}{Visualisierung des ersten normalen Ventrikelsystems von der Seite mithilfe von Unity}
\label{fig:unity_s}
\end{minipage}
\begin{minipage}[c]{0.49\textwidth}
\includegraphics[width=\textwidth]{Logos/Normal1_Unity/Unten.PNG}
\captionof{figure}{Visualisierung des zweiten normalen Ventrikelsystems von Unten mithilfe von Unity}
\label{fig:unity_u}
\end{minipage}


\begin{minipage}[c]{0.49\textwidth}
\includegraphics[width=\textwidth]{Logos/Normal1_MITK/Oben.PNG}
\captionof{figure}{Visualisierung des ersten normalen Ventrikelsystems von Oben mithilfe von MITK}
\label{fig:mitk_o}
\end{minipage}
\begin{minipage}[c]{0.49\textwidth}
\includegraphics[width=\textwidth]{Logos/Normal1_MITK/Schraeg_Vorne.PNG}
\captionof{figure}{Visualisierung des ersten normalen Ventrikelsystems von Vorne mithilfe von MITK}
\label{fig:mitk_v}
\end{minipage}


\subsection{Nutzerstudie}


Im Rahmen der Evaluation der Benutzerfreundlichkeit des Verfahrens wurde eine kleine Nutzerstudie mit ... Teilnehmern durchgeführt. Bei dieser wurden den Probanden zunächst der Ablauf und die vom Benutzer erforderlichen Schritte um eine Visualisierung des Ventrikelsystems zu erhalten durch eine Vorführung durch den Interviewer gezeigt. Anschließend mussten die Teilnehmer selbst das eben gelernte anwenden und das Programm selber ausführen. Dabei bekamen sie, wenn sie nicht weiterwussten, Hilfe vom Versuchsleiter. Als Abschluss füllten die Probanden einen NASA-TLX Bogen zu der Aufgabe aus. Die durchschnittlichen Ergebnisse der einzelnen Kategorien wird in \autoref{tab:ergebnis_nasa} gezeigt.


\begin{table}[h]
\centering
\resizebox{\columnwidth}{!}{
 \begin{tabular}{| c | c | c | c |}
  \hline
  Kategorie & Gewichtung & Klicks & Wichtung \\ \hline
  Geistige Anforderung & 0&0 &0 \\ \hline
  Körperliche Anforderung & 0& 0& 0\\ \hline
  Zeitliche Anforderung &0 &0 &0 \\ \hline
  Leistung &0 &0 & 0\\ \hline
  Anstrengung &0 & 0& 0\\ \hline
  Frustration &0 &0 & 0\\ \hline 
 \end{tabular}
 }
\caption{Durchschnittlichen Ergebnisse des NASA-TLX Bogens}
\label{tab:ergebnis_nasa}
\end{table}


Der durchschnittliche Wert für die Gesamtbeanspruchung lag bei ... Personen ohne Programmiererfahrung ...

 
Trotz der Schwierigkeiten,  gaben die Probanden an, dass sie die Aufgabe mit einer guten ausführlichen Dokumentation auch alleine ohne Hilfe hinbekommen würden.


\subsection{Berechnungszeit}

Die folgenden Zeitmessungen wurde alle auf einem Computer mit einem 3.70GHz  Intel Core(TM) i7-8700K CPU mit 32GB RAM ausgeführt.
\newline
Um die Berechnungszeit des Systems evaluieren zu können, wurde die Kalkulation des gesamten Clusteringverfahrens sowie die Berechnung des LH-Histogramms auf drei Volumen verschiedener Größen durchgeführt. Damit der Vergleich nicht von unterschiedlichen Volumendaten verfälscht wird, wurden alle Volumen aus dem gleichen CT-Datensatz generiert. Dabei wurde die Originalgröße mit einer Auflösung von 512x201x512 Pixeln mithilfe des Resamplemoduls des Helpers verkleinert. Die beiden anderen Volumengrößen haben dabei die  Hälfte, mit 256x101x256 Pixeln, und Dreiviertel, mit 384x151x384 Pixeln, der Auflösungen des Originalvolumens. Es war geplant, dass noch ein viertes Volumen zum Vergleich hinzugezogen wird.Jedoch war es aus einem unbekannten Fehler leider nicht möglich die beiden Berechnungen mit dem gevierteltem Originalvolumen durchzuführen. Des Weiteren funktioniert für beim Originalvolumen lediglich die Berechnung des LH-Histogramms. Die Kalkulation des Clusteringverfahrens war nicht möglich, vermutlich aus dem Grund, dass bei dieser hohen Auflösung es zu viele Daten für die aktuelle Implementierung zu berechnen gibt.
\newline
Die Berechnungszeit hängt stark von der Größe des Eingabevolumens ab. Die ist in \autoref{tab:ueberblick_zeit} sehr gut zu erkennen. Diese zeigt einen Überblick über die ungefähren Berechnungszeiten der verschiedenen Volumengrößen.


\begin{table}[h]
\centering
\resizebox{\columnwidth}{!}{
 \begin{tabular}{| c | c | c | c |}
  \hline
  Volumengröße & LH-Histogramm $[s]$ & Komplettes Verfahren $[s]$ \\ \hline
  Halbes Volumen (256x101x256)  & 30 &  50	\\ \hline
  Dreiviertel Volumen (384x151x384)  & 90 &  380	\\ \hline
  Ganzes Volumen (512x201x512) & 225 & -	\\ \hline
 \end{tabular}
 }
\caption{Überblick über die Berechnungszeiten der verschiedenen Volumengrößen}
\label{tab:ueberblick_zeit}
\end{table}


Dabei ist wichtig zu beachten, dass die Zeit zur Berechnung der LH-Histogramme die gleiche Zeit wie die Kalkulation der LH-Werte im gesamten Verfahren benötigt. Die Berechnungsdauer der Gradienten ist hierbei zirka doppelt so lange wie die der LH-Werte. Wird die Berechnungszeit des Histogramms von der Kalkulationszeit des gesamten Verfahrens abgezogen, kommt die Zeit, die die beiden Clusteringschritte benötigen heraus.
\newline
Eine interessante Beobachtung hierbei ist, dass die Berechnung der LH-Histogramme abhängig von der Anzahl der Pixel gesehen in etwa gleich schnell abläuft. Das halbe Volumen hat eine Gesamtpixelzahl von ungefähr 6,6 Millionen, das dreiviertel Volumen von zirka 22,2 Millionen und das Original von grob 52,6 Millionen Pixeln. Wird die Anzahl an Pixeln die pro Sekunde bei der LH-Wert Berechnung bearbeitet werden für diese drei Volumen berechnet, so ist zu beobachten, dass keine großen Unterschied zwischen den Zeiten existiert. Das Halbe bearbeitet etwa 220 Tausend, das Dreiviertel ungefähr 247 Tausend und das Ganze 234 Tausend Pixel pro Sekunde. Der kleine Unterschied in der Rate lässt sich einerseits durch Volumen unabhängige Berechnungen, und andererseits durch Messfehler erklären. Folglich kann die Aussage getroffen werden, dass die Berechnungszeit der LH-Werte bei dieser Implementierung in etwa linear mit der Anzahl an Eingabepixeln wächst.
\newline
Auf der anderen Seite ist jedoch auch zu erkennen, dass die beiden Clusteringschritte mit zunehmender Eingabegröße deutlich langsamer werden. Das Clustering des halben Volumens dauerte 20 Sekunden und hat damit eine Verarbeitungsrate von zirka 330 Tausend Pixeln pro Sekunde. Hingegen dauert es beim dreiviertel Volumen 290 Sekunden und erreicht damit gerade einmal einen Rate von 76 Tausend Pixeln pro Sekunde. Es benötigt also 14,5 mal so viel Zeit für die 3,3 fache Anzahl an Pixeln.















































%% ==============================
\chapter{\iflanguage{ngerman}{Fazit}{Discussion}}
\label{sec:discussion}
%% ==============================

\dots



%% ==============================
\chapter{\iflanguage{ngerman}{Ausblick}{Conclusion}}
\label{sec:conclusion}
%% ==============================



Die Implementierung eines Clusteringbasierten Verfahren kam in dieser Arbeit zu einem Erfolg, jedoch gibt es noch sehr viele Verbesserungs- und Erweiterungsmöglichkeiten. Diese werden im folgenden Kapitel vorgestellt.


Die Visualisierung des Ventrikelsystems funktionierte bei nur wenigen Datensätzen. Bei den restlichen wurde das System aus verschiedenen Gründen nicht erkannt.
\newline
Es gab Probleme, wenn einerseits das Ventrikelsystem zu klein oder verformt war, oder andererseits die Hirnmasse im Kopf und um den Ventrikel herum andere Intensitätswerte als üblich hatten.
\newline
Bei normalen Ventrikel funktionierte es teilweise auch nicht. Diese hohe Inkonsistenz gilt es zu beheben. Dazu können verschiedene Veränderungen vorgenommen werden.
\newline
Da das Verfahren oft daran scheiterte, die Kanten von Ventrikeln zu erkennen, da sie zu klein oder der Unterschied zur Umgebung zu gering ist, wäre es eine Möglichkeit, einen noch kleinere Bandweite und damit ein noch genaueres Clustern über dem LH-Raum zu implementieren.
\newline
Dadurch würden jedoch noch mehr Cluster entstehen, was die Benutzung noch schwieriger macht.
\newline
Dies könnte jedoch behoben werden, indem einerseits Cluster die am Rande des Volumens liegen weggelassen werden. Da das Ventrikelsystem in der Mitte des Kopfes liegt, ist es klar, dass diese Cluster keine Teile eines Ventrikels enthalten können.
\newline
Andererseits könnte der LH-Wertebereich über dem geclustert wird noch kleiner gemacht werden. Wie in \autoref{fig:unity_s} und \autoref{fig:unity_u} zu sehen ist, werden die Seitenventrikel in einem Intensitätsbereich von 1025 bis 1030 bereits erkannt.
\newline
Allgemein würde ein weiteres Anpassen der Parameter auf das Ventrikelsystem zu besseren Ergebnissen führen.


Ein weiteres Problem war die Benutzerfreundlichkeit. Ein Anwender muss einen genauen Ablauf befolgen um zu einem Ergebnis zu kommen, indem er erst mithilfe der Konsole Daten laden, falls sie noch im .nrrd Format vorliegen formatieren, die Cluster erstellen lassen, diese in Unity laden, die passenden Cluster manuell herausfinden, und schließlich mit einem weiteren Befehl über die Kommandozeile zu einem Ergebnis verschmelzen lassen. Dieser Ablauf ist wenig Intuitiv und kann wie die Nutzerstudie aus Kapitel 6 gezeigt hat für Menschen ohne Programmiererfahrung zu einer Hürde werden.
\newline
Dies könnte an vielen Stellen verbessert werden. Beispielsweise können die Befehle der Konsole vereinfacht werden, sodass der Benutzer das notwendige $u$ sowie das Suffix .bin.txt nicht mehr angeben muss, da das Programm automatisch das Volumen umwandelt und im richtigen Dateiformat speichert.
\newline
Eine andere Idee wäre es, die zwei Programme zu einem zu verschmelzen, sodass der Anwender direkt in Unity das Volumen laden und die IDs berechnen lassen könnte.
\newline
Des Weiteren wäre es dann möglich, dass die Cluster direkt in Unity angezeigt, ausgewählt und verschmolzen werden könnten. Dadurch bestünde auch die Möglichkeit zwischen der Cluster und der finalen Ansicht hin und her zu wechseln.
\newline
Der Anwender muss sich dadurch nicht mehr mit Dateipfaden dem Speichern und Einlesen von Dateien im richtigen Format beschäftigen.
\newline
Weiterhin könnte Nutzung durch das Erstellen einer GUI intuitiver gemacht werden, damit keine Eingabe von Befehlen in eine Kommandozeile mehr nötig ist.


Ein Problem, dass auch vom Arzt erkannt wurde, ist die nicht glatte Darstellung im Ergebnis, sowie die vielen existierenden Ausreißer. Dies macht die Visualisierung teilweise sehr ungenau.
\newline
Dies lies sich lösen, indem auf das Endergebnis weitere Algorithmen zur Verbesserung der Segmentierung angewendet werden.
\newline
Beispielsweise könnten mit Dilatations- und Erosionsfiltern die Oberfläche geschlossen und kleine Ausreißer eliminiert werden.
\newline
Eine weitere Verbesserung hierbei wäre, einen Regiongrowingalgorithmus am Ende des Verfahrens anzuwenden. Dieser könnte das Ergebnis deutlich verbessern, da dadurch ein vollständigeres Bild des Ventrikelsystems entstehen und eventuell sogar weitere Ventrikel, wie zum Beispiel das Dritte und Vierte, erkannt werden könnten.


Weiterhin könnte an vermutlich mehreren Stellen des aktuellen Codes kleine Optimierungen vorgenommen werden.
\newline
Beispielsweise die Verbesserung der Suche nach Schleifen beim Berechnen der High-Werte, die im Kapitel \nameref{sec:implementation} beschrieben wurde.
\newline
Oder es könnte beim räumlichen Clustern ein Verschmelzen der im gleichen LH-Cluster gefundenen räumlichen Cluster hinzugefügt werden.
\newline
Dies sind nur zwei Beispiele und es gibt vermutlich mehrere Stellen an denen diese kleinen Verbesserungen vorgenommen werden könnten.
\newline
Diese Änderungen hätten möglicherweise nur geringe Auswirkungen, sie  könnten jedoch nützlich sein, um ein optimales Ergebnis zu erreichen.


Ein weiterer Punkt für Verbesserungen ist das Beheben der Bugs. Der Grund für das scheitern der Berechnung des Verfahrens auf dem ganzen, sowie auf den viertel Volumen muss gefunden und behoben werden.


Des Weiteren könnte getestet werden ob die Berechnungszeit noch weiter zu verbessern ist. Aktuell laufen alle Kalkulationen auf der CPU. Die Clusteringschritte könnte jedoch auf der GPU deutlich schneller berechnet werden.




%% --------------------
%% |   Bibliography   |
%% --------------------
\Bibliography{Bibliography/thesis}

%% ----------------
%% |   Appendix   |
%% ----------------
% \cleardoublepage
%% appendix.tex
%%

%% ==============================
\Appendix
\label{ch:Appendix}
%% ==============================



\section{Technische Doku über die Benutzung von den MITK Segmentierungstools}

Als erstes muss in den DICOM Ordner der Ordner mit ungefähr 200 Elementen gefunden werden.

\begin{figure}[H] 
\centering 
\includegraphics[width=0.7\textwidth]{Logos/MITK_Doku/1.PNG}
\caption{1} 
\label{fig:eins} 
\end{figure}

Anschließend kann eine der Dateien des Ordners via drag and drop in den \textit{Data Manager} gezogen werden und das Fenster das erscheint bestätigt werden.
\newline
In der Mitte sind die verschiedenen Schnittbilder zu sehen, durch die mit dem Mausrad durchgegangen werden kann.
\newline
Die Zahlen von 1 bis 5 markieren verschiedenen Tools. Werden diese angeklickt, öffnet sich an der Seite oder unter den Schnittbildern das jeweilige Tool. Um ein Tool auf ein Volumen anzuweden, muss dieses immer im \textit{Data Manager} ausgewählt sein.
\newline
Mit dem Regler am rechten Rand der Schnittbilder kann die Darstellung der Grauwerte eingestellt werden.

\begin{figure}[H] 
\centering 
\includegraphics[width=\textwidth]{Logos/MITK_Doku/2.PNG}
\caption{2} 
\label{fig:zwei} 
\end{figure}

Das Tool mit der Nummer 1 macht es möglich eine Transferfunktion zu definieren. Anhand der Intensitätswerte, den Punkten über dem Diagramm und dem Farbstrahl darunter kann die Transferfunktion angepasst werden. Ist der Haken bei \textit{Volumerendering} gesetzt, so wird die Visualisierung im Fenster links unten angezeigt.

\begin{figure}[H] 
\centering 
\includegraphics[width=0.7\textwidth]{Logos/MITK_Doku/3.PNG}
\caption{3} 
\label{fig:drei} 
\end{figure}

Das Tool mit der Nummer 2 zeigt Statistiken zum Volumen an, wie zum Beispiel Verteilung der Intensitätswerte, der Median dieser etc.

\begin{figure}[H] 
\centering 
\includegraphics[width=0.4\textwidth]{Logos/MITK_Doku/4.PNG}
\caption{4} 
\label{fig:vier} 
\end{figure}

Das Tool Nummer 3  bietet verschiedene 2D und 3D Segmentierungstools an. Zunächst muss jedoch eine neue Segmentierung über den rot markierten Knopf erstellt werden.

\begin{figure}[H]
\centering 
\includegraphics[width=0.7\textwidth]{Logos/MITK_Doku/5.PNG}
\caption{5} 
\label{fig:fuenf} 
\end{figure}

Wurde dies getan, kann das 3D Tool \textit{Region Growing 3D} ausgewählt werden.

\begin{figure}[H] 
\centering 
\includegraphics[width=0.7\textwidth]{Logos/MITK_Doku/6.PNG}
\caption{6} 
\label{fig:sechs} 
\end{figure}

Anschließend kann mit Shift + Linke Maustaste eine Punkt im Volumen ausgewählt werden. Danach muss mit  \textit{Run Segmentation} die Segmentierung ausgeführt werden.

\begin{figure}[H] 
\centering 
\includegraphics[width=0.7\textwidth]{Logos/MITK_Doku/7.PNG}
\caption{7} 
\label{fig:sieben} 
\end{figure}

Nun kann der Threshhold angepasst und das Region Growing mit dem \textit{Adapt Region Growing} Regler angepasst werden. Wird das gezielte Ergebnis erwünscht, kann es mit \textit{Confirm Segmentation} bestätigt werden.

\begin{figure}[H] 
\centering 
\includegraphics[width=0.7\textwidth]{Logos/MITK_Doku/8.PNG}
\caption{8} 
\label{fig:acht} 
\end{figure}

Mit Tool Nummer 4 kann das Ergebnis mit verschiedenen Operationen verbessert werden. Hierbei wird oft \textit{Closing} benutzt. Anschließend kann mit Rechtsklick auf die Segmentierung im \textit{Data Manager} der Befehl \textit{Create smoothed polygon model} ausgeführt werden

\begin{figure}[H] 
\centering 
\includegraphics[width=0.7\textwidth]{Logos/MITK_Doku/9.PNG}
\caption{9} 
\label{fig:neun} 
\end{figure}

Nachdem alle diese Schritte befolgt wurden ist im linken Unteren Kasten die Segmentierung sehen. Mit Tool Nummer 5 kann die Darstellung der Segmentierung verändert werden.

\begin{figure}[H] 
\centering 
\includegraphics[width=0.7\textwidth]{Logos/MITK_Doku/10.PNG}
\caption{10} 
\label{fig:zehn} 
\end{figure}

\end{document}
