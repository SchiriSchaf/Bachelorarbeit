\documentclass{article}
\usepackage{ucs}
\usepackage[utf8x]{inputenc}
\usepackage[T1]{fontenc}
\usepackage[ngerman]{babel}
\usepackage{pdfpages}
\usepackage{graphicx}
\usepackage{hyperref}
\usepackage{cite}
\usepackage{pdfpages}
\usepackage{rotating}
%\usepackage{tocstyle}
%\documentclass[german]{seminar}
\setlength{\parindent}{0pt} %\ neuer Absatz ohne Einrücken
\title{
Exposé:\\
 \textbf{"Implementierung von Transferfunktionen zur Visualisierung von Volumen Modellen auf einer AR-Brille"}
}


%\date{26.04.2018}
\author{Lukas Diewald, uodzo@student.kit.edu}

\begin{document}
\maketitle

\clearpage
\section{Motivation}
Computergestütze Verfahren werden in der Medizin immer wichtiger. Sie erleichtern dem Arzt seine Arbeit und senken die Wahrscheinlichkeit, dass bei einem Eingriffen oder einer Behandlungen ein Fehler gemacht wird.

\section{Problemstellung}
Eine Ventrikelpunktion ist ein operativer Eingriff am Gehirn, der von einem Neurochirugen durchgeführt wird. Er dient zur Entnahme von Nervenflüssigkeit, die im Anschluss untersucht werden kann. Der Chirug führt hierbei eine Bohrlochtrepantation am Kocherpunkt durch, also er bohrt an einem speziellen Punkt ein Loch in die Schädeldecke. Der Eingriff muss möglichst genau erflogen, um ungewollte Schäden am Gehirn zu vermeiden.
\newline
Ziel dieser Bachelorarbeit ist, das Volumenmodell eines Gehirns mithilfe einer AR-Brille zu visualisieren. Weiterhin soll es mithilfe von Transferfunktionen möglich sein einzelne Gebiete des Gehirns hervorzuheben.

\section{State of the art}

\section{Vorgehen}

\section{Zeitliche Planung}

\section{Literatur}



\end{document}